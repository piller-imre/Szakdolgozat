%Az összefoglaló fejezet
\chapter*{Adathordozó használati útmutató}
\addcontentsline{toc}{chapter}{Adathordozó használati útmutató}

%Ebben a fejezetben kell megadnunk, hogy a szakdolgozathoz mellékelt adathordozót (pl. CD) hogyan lehet elérni, milyen strukturát követ. Minimum 1 maximum 4 oldal a terjedelem. Lehet benne több alszakasz is. A fejezet címe nem módosítható, hasonlóan a következõ részhez (Irodalomjegyzék).

% TODO: Mindenhova egész mondatok kellenének majd.

\noindent A szakdolgozatomhoz mellékelt adathordozó eszközön található jegyzékek és fájlok az alábbiak.

\bigskip

\noindent \texttt{$\backslash$hasznalati\_utmutato.pdf}

\medskip

Felhasználói dokumentáció, amely leírja, hogy az adathordozón található fájlok hogyan használhatóak.

\bigskip

\noindent \texttt{$\backslash$dolgozat.pdf}

\medskip

Szakdolgozat teljes szövege.

\bigskip

\noindent \texttt{$\backslash$FuttathatoValtozat}

\medskip

Futtatható változatot tartalmazó jegyzék az összes szükséges bináris állománnyal.

\bigskip

\noindent \texttt{$\backslash$FuttathatoValtozat$\backslash$Szakdolgozat.exe}

\medskip

A szakdolgozathoz készített futtatható állomány.

\bigskip

\noindent \texttt{$\backslash$KodForrasszovege}

\medskip

Kód forrásszövegét tartalmazó jegyzék

\bigskip

\noindent \texttt{$\backslash$Kepek}

\medskip

Az alkalmazásról készített képek jegyzéke.

\bigskip

\noindent \texttt{$\backslash$Kepek$\backslash$OsztályokHierarchiája.jpg}

\medskip

Az osztályok hierarchiáját ábrázoló kép.
