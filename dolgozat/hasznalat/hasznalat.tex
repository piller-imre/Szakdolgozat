%Az összefoglaló fejezet
\chapter*{Adathordozó használati útmutató}
\addcontentsline{toc}{chapter}{Adathordozó használati útmutató}

%Ebben a fejezetben kell megadnunk, hogy a szakdolgozathoz mellékelt adathordozót (pl. CD) hogyan lehet elérni, milyen strukturát követ. Minimum 1 maximum 4 oldal a terjedelem. Lehet benne több alszakasz is. A fejezet címe nem módosítható, hasonlóan a következõ részhez (Irodalomjegyzék).

\noindent A szakdolgozatomhoz mellékelt adathordozó eszközön található adatok struktúrája:

\begin{center}
  \begin{tabular}{ | c | c | }
    \hline
      Adat megnevezése & Elérési útvonal \\ \hline
      Felhasználói dokumentáció &  /FelhasznaloiDokumentacio\\ \hline
      Futtatható változat & /FuttathatoValtozat  \\ \hline
      Futtatható változat futtatható állománya & /FuttathatoValtozat/Szakdolgozat.exe  \\ \hline
      Kód forrásszövege & /KodForrasszovege  \\ \hline
      Szakdolgozat teljes szövege & /SzakdolgozatTeljesSzovege  \\ \hline
  \end{tabular}
\end{center}

