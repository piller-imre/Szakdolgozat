\Chapter{Térkép generálása}

\section{Metrikák}

A hexagonok és a négyzetek esetén az útkeresés csak egy ponton különbözik (szomszédok száma) ezért az egyszerûség kedvért csak a négyzethálónál mutatom be

hexagon
offset coord
pointy top

Bemenetként a térkép fõ jellemzõit, a generált térképpel szembeni elvárásokat kapja paraméterezésként.

A térkép egy több lépcsõs folyamat végén fog elkészülni. 
Az elsõ fázisban a nagyobb területi egységek (szigetek, folyók, ...) körvonalazódása történik. 
A második lépésben a térkép alapegységeinek (tile) a konkretizálása történik meg, azaz leképzõdik egy hexagon háló. 
Harmadik lépésként minden egyes tile-hoz hozzárendeli az algoritmus a megfelelõ textúrát a területi egységek alapján. 
A negyedik fázisban városokat, településeket, falakat hoz létre a program. 
Ötödik lépésként a növényzetet (fák, bokrok) generálja majd le. 
A hatodik lépésben a dekorálás jön, ahol az évszakoknak és a különbözõ természeti hatásoknak megfelelõen módosulhat a textúra.

Elsõ lépésként szükségünk lesz egy csempére (tile), ami az alapelem lesz a térképen. Erre a célra én egy 3D-s hexagon modellt használtam különbözõ textúrákkal.

Ezen kívül szükség van a generálni kívánt térkép méreteire (szélesség, magasság).

