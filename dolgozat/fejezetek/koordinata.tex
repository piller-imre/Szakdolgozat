\Chapter{Koordináta rendszerek}

Négyzet
Négyzetrács esetében egy derékszögû koordinátarendszer használata a legkézenfekvõbb. 

A derékszögû koordináta-rendszert két egymásra merõleges számegyenes alkotja. Az egyeneseket koordinátatengelyeknek, metszéspontjukat kezdõpontnak, origónak nevezzük. Az origóhoz mindkét számegyenesen a 0-t rendeljük hozzá. A vízszintes tengely az x (abszcissza) tengely, a függõleges az y (ordináta) tengely.

A koordináta-rendszer segítségével a sík bármely P pontjának a helyzete két jelzõszám (koordináta) segítségével egyértelmûen meghatározható. A pont helyzetét a két tengelytõl mért elõjeles távolságával határozzuk meg. A pontnak a tengelyektõl mért elõjeles távolságai a pont koordinátái (jelzõszámai). Az elõjelek a számegyenesek segítségével adhatók meg. A jelzõszámokat, a pont neve után zárójelben adjuk meg: P(x;y).
Hexagon
A hexagonháló esetében többfajta megközelítés is szóbajöhet, most ezek közül fogok néhányat ismertetni. 
Eltolásos koordináta rendszer (Offset coordinates)

	A leggyakoribb megközelítés az eltolásos módszer, ami kisebb eltérésektõl eltekintve gyakorlatilag megegyezik a négyzet koordináta rendszerrel. 

Ha megnézzük a lenti képet, akkor láthatjuk, hogy a hexagonhálóhoz hasonló hatást kapunk, ha a négyzethálóban minden páros/páratlan sort/oszlopot eltolunk.

Eltolható a páros és a páratlan oszlop/sor is. Mivel kétféleképpen is állhatnak a hexagonok, ezért 4 fajta variáció érhetõ el összesen.
		
Kocka koordináta rendszer (Cube coordinates)

	A kocka koordináta rendszerben az eddig megszokottakkal ellentétben nem kettõ, hanem három fõ tengely van.

Tengely koordináta rendszer (Axial coordinates)
IV. Ábrázolás
Négyzet

B. Hexagon

A hexagonok alapvetõen kétféleképpen állhatnak. 
Az egyik lehetõség, hogy az egyik csúcs van felül;
a másik lehetõség, hogy az egyik oldal van felül. 

A következõ lépésként vizsgáljuk meg, hogy hogyan tudjuk egymás mellé elhelyezni a hexagonokat.

A felül hegyes elrendezés esetén vízszintesen a hexagon szélességével, függõlegesen pedig a hexagon magasságának a 3 -vel kell eltolni következõ hexagont. Ezen kívül minden páros/páratlan sort vízszintesen a szélesség felével kell még eltolni.

A felül lapos elrendezés esetén vízszintesen az egymás melletti hexagonok közötti távolság a hexagon szélességének 3 része. Függõlegesen minden páros/páratlan oszlopot a magasság felével kell eltolni.
