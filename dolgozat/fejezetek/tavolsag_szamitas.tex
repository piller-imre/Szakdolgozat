\chapter{Távolság számítás}

\section{Kocka koordináta-rendszer}

A kocka koordináta-rendszerben, minden hexagon egy kockának felel meg a 3D térben. A szomszédos hexagonok 1 távolságra vannak egymástól a hexagonháló esetén, viszont 2 távolságra vannak egymástól a kocka koordináta-rendszerben. Ezáltal téve egyszerűvé a távolságszámítást. A négyzet koordináta-rendszer esetén a Manhattan-távolság képlet a következő: $abs(dx) + abs(dy)$. A kocka koordináta-rendszerben pedig a Manhattan-távolság $abs(dx) + abs(dy) + abs(dz)$. Ezekből következik, hogy a hexagon rács esetén a Manhattan-távolság képlet a kockarács esetén fennálló távolság fele.
\newline
\newline Pseudo kóddal:
\begin{verbatim}
function cube_distance(a, b):
    return (abs(a.x - b.x) + abs(a.y - b.y) + abs(a.z - b.z)) / 2
\end{verbatim}    

$$
d_{\text{cube}}(a, b) =
\dfrac{|a_x - b_x| + |a_y - b_y| + |a_z - b_z|}{2}
$$

\noindent Egy ezzel egyenértékű megközelítés az is, ha megfigyeljük azt, hogy a három koordináta közül az egyiknek mindenképpen a a másik kettő összegének kell, hogy legyen, ekkor az az egy lesz a távolság. 
\newline
\newline Pseudo kóddal:
\begin{verbatim}
function cube_distance(a, b):
    return max(abs(a.x - b.x), abs(a.y - b.y), abs(a.z - b.z))
\end{verbatim}

$$
d_{\text{cube}}(a, b) =
\max(
|a_x - b_x|, |a_y - b_y|, |a_z - b_z|
)
$$

\noindent Az, hogy melyik algoritmust váltjuk az szituáció és egyén függő, de az eredmény ugyanaz.

\section{Tengelyes koordináta-rendszer}

A tengelyes koordináta-rendszerben a számításokhoz szükséges három koordinátából kettőt ismerünk, ezért konvertálnunk kell majd a kocka koordináta-rendszerre.
\newline
\newline Pseudo kóddal:
\begin{verbatim}
function hex_distance(a, b):
    var ac = axial_to_cube(a)
    var bc = axial_to_cube(b)
    return cube_distance(ac, bc)
\end{verbatim}    

\noindent Egy másik módszer arra az esetre, ha egy függvényben szeretnénk megoldani:
\begin{verbatim} 
function hex_distance(a, b):
    return (abs(a.q - b.q) 
          + abs(a.q + a.r - b.q - b.r)
          + abs(a.r - b.r)) / 2
\end{verbatim}          

\noindent Mivel a tengelyes koordináta-rendszerben is a távolságokat kockarács esetén alkalmazható Manhattan-távolság számító algoritmussal számítjuk ki, ezért valamilyen módon arra kell visszavezetni.

\section{Eltolásos koordináta-rendszer}

Az eltolásos koordináta-rendszer esetén is azt a megoldást fogjuk használni amit a tengelyes koordináta-rendszer esetén is alkalmaztunk. Átalakítjuk a koordinátákat a kocka koordináta-rendszerre.
\newline
\newline Pseudo kód:
\begin{verbatim} 
function offset_distance(a, b):
    var ac = offset_to_cube(a)
    var bc = offset_to_cube(b)
    return cube_distance(ac, bc)
\end{verbatim}  
