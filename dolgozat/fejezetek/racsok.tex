\Chapter{Rácsok összehasonlítása}

Legyen szó társasjátékról vagy számítógépes játékról, az egyik leggyakrabban használt rács a négyzetrács. Egyszerû, könnyen kezelhetõ és jól illeszthetõ a számítógép kijelzõjére. A cellák pozícióit a Descartes-féle derékszögû koordináta-rendszer (x, y) segítségével határozhatjuk meg. Kirajzolásához ismernünk kell a cellai méreteit (szélesség, magasság) illetve a rács méreteit (oszlopok, sorok száma). A nyilvántartáshoz ismernünk kell a viszonyítási pontot (origó), illetve tudnunk kell az objektum pozicióját (x, y).

Könnyen kezelhetõsége mellett viszont van egy nagy hátránya:
Egy négyzetnek nyolc szomszédja van. Oldalain keresztül vízszintesen, valamint függõlegesen 2-2 szomszédja érhetõ el. További négy szomszédja átlósan található meg. A problémára akkor figyelünk fel, amikor megvizsgáljuk a távolságot a különbözõ szomszédok között. A vizsgálathoz tegyük fel, hogy az oldalak hossza 1. Ha a négyzetek középpontjához viszonyítunk, akkor a függõlegesen és a vízszintesen lévõ szomszédok távolsága 1, míg az átlósan lévõk távolsága ?2 (Pitagorasz-tétel).

A két fajta szomszéd közötti különbség miatt vetõdnek fel bizonyos kérdések:
Hogyan kezeljük az átlós mozgást? 
Egyáltalán engedélyezzük-e az átlós mozgást? 

A problémára több megoldás is létezik:

Nem alkalmazunk átlós mozgást. Ez a legegyszerûbb megoldás, amit az egyszerûsége miatt gyakranhasználnak
Egy kevésbé elterjedt megoldás, hogy maradunk a négyzeteknél, viszont minden második sort/oszlopot eltolunk az oldal szélességének/hosszának a felével. Ekkor az összes szomszéd hasonló távolságra kerül.

3. A leggyakoribb megoldás a hexagonok használata a négyzetek helyett. A négyzethez hasonlítva a hatszögnek csak hat szomszédja van (nyolc helyett). Ezek közül mindegyik oldal szomszéd, és nincs olyan szomszédja ami a sarkokhoz esne. Ezáltal minden szomszéd egyenlõen 1 távolságra van.
B. Rácsok felhasználása

Alapvetõen mindkét rácsnak megvan a maga helye a játékokban. 

Mivel a beltéri helyszínek (szobák) és az azon belüli elemek (bútor) általában téglalap alakúak, praktikusabb a négyzetrács használata. A négyzetrács a falakhoz tökéletesen illeszkedik, ugyanakkor a hexagonok esetében problémák merülnek fel. Ugyanis a hexagonok nem fognak szabályosan illeszkedni a falak mentén. Erre kétfajta megoldás létezik: a fal menti hexagonokat elvághatjuk, vagy másik megoldás, ha nem töltjük ki a fennmaradó helyeket. Egyik megoldással sem lehetünk maradéktalanul elégedettek,
ha elvágjuk a hexagonokat. 

Akkor azok hogyan viselkedjenek? 
Lehessen-e rálépni?
Ha nem lehet rálépni akkor miért van?

Ha csak kihagyjuk a széleken a hexagonokat amik nem férnek el az a felhasználó számára furcsa összhatást nyújthat.

Ha mindenképpen hexagonokat  szeretnénk használni kis méretû térképen (pl: 8x20), akkor lehetõleg ne egy zárt szobában alkalmazzuk hanem szabad téren (pl: mezõ, erdõ, tenger part) vagy próbáljuk a hexagon rács széleihez igazítani a környezetet.

Kültéren, falak hiányában ezek a problémák nem merülnek fel. Emiatt, valamint a négyzetrács átlós mozgásával kapcsolatos problémák miatt elõnyösebb a hexagonháló használata ezekben az esetekben.

