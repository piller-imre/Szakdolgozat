\chapter{Térkép tárolási probléma}

Az egyik leggyakoribb probléma a tengelyes koordináta rendszerrel, hogy téglalap alakú térkép esetén elpazarolt helyek lesznek a mátrixban. Ez a fő érv az eltolásos koordináta rendszer mellett. Azonban az összes korábban említett koordináta rendszer elpazarolt helyekhez vezet, amikor háromszög vagy hatszög alakú térképet használunk.
\newline
\newline mellékletben képek
\newline
\newline Vegyük észre a képeken, hogy az elpazarolt hely a sorok bal és jobb szélén jelentkezik (kivéve a rombusz esetén). Ez három stratégiát ad nekünk a tárolásra:
\newline
\newline Hagyjuk figyelmen kívül a problémát. Használjunk mátrixot a tárolásra és használjunk valamilyen speciális jelzőt a nem létező mezőkre. Nem éri meg ennél komplikáltabb megoldást.
Használjunk valamilyen listát a mezőkről a mátrix helyett. Ezáltal lehetőségünk lesz szabálytalan formájú térképek készítésére, beleértve azt is, hogy legyen egy lyuk a közepén. A rács osztályból getter/setter metódusok segítségével könnyen el lehet érni. (pl: $Grid(Tile(x,y))$);
Csúsztassuk el a sorokat úgy, hogy bal oldalt ne legyen “üres” hely. Néhány nyelvben a 2D tömb az egy tömbökből álló tömb, ilyen esetekben a tömböknek nem kell egyforma hosszúaknak lennie, így eltüntethető a felesleg a jobb oldalról is.
\newline
\newline Ahhoz, hogy ilyen különböző konvex formájú térképet tároljunk szükségünk lesz egy plusz tömbre az “első oszlopok” tárolásához. Amikor hozzá akarunk férni a hexagonhoz az q,r koordinátákon akkor az $array[r][q - első_oszlop[r]]$ kell használnunk inkább. A rács osztályból getter/setter metódusok segítségével könnyen el lehet érni.
\newline
\newline Ha a térkép fix formájú, akkor az első oszlop menet közben is számítható ahelyett, hogy tárolnánk.
\newline
\newline A téglalap alakú térképek esetén, $első_oszlop[r] == -floor(r/2)$, és ezáltal érjük el a hexagonokat úgy, hogy $grid[r][q + r/2]$, ami ekvivalens azzal mintha konvertálnánk eltolásos koordináta rendszerbe.
A háromszög alakú térképek esetén, $első_oszlop[r] == 0$, és ezáltal érjük el a hexagonokat úgy, hogy $grid[r][q]$. Abban az esetben, ha a háromszög nem a képen látható módon áll, hanem csúccsal felfelé, akkor $grid[r][q + r]$.
A hexagon formájú $N$ sugarú térképek, ahol $N = max(abs(x), abs(y), abs(z))$, és $első_oszlop[r] == -N - min(0, r)$. Viszont hogyha $r < 0$ értékkel kezdünk, akkor el kell tolnunk a sorokat és úgy érhetjük el $grid[r + N][q + N + min(0, r)]$.
A rombusz formájú térkép esetén minden tökéletesen egyezik ezért egyszerűen csak $grid[r][q]$ formát használjuk.
