\chapter{Összefoglalás}

A szakdolgozatom célja egy olyan procedurális generálási mód megvalósítása volt, ami változatos térképet generál izometrikus grafikájú játékokhoz. Az elméleti ismeretek bemutatása után az általam preferált technikák kiválasztásának indoklása és az implementálásukhoz szükséges részletek kifejtése történt meg. Végezetül a program implementációjának bemutatása következett. A szakdolgozatom során igyekeztem ábrákkal segíteni az általam kevésbé egyértelműnek tűnő részeket. 

\bigskip

A szakdolgozatom készítése során sikerült létrehoznom egy olyan algoritmust amely a megadott bermeneti értkek alapján állít elő "véletlenszerű" térképet. A programomnak vannak még hiányosságai, amely további fejlesztést igényel.

\bigskip

Az alkalmazás további fejlesztésére különféle lehetőségek vannak. Az alábbiakban felvetés szintjén szerepel ezek közül néhány.
% TODO: Jó listának, csak mondatokba kellene szedni.
\begin{itemize}
\item Teljesítménybeli optimalizálás
\item További épületek/növények/utak/hidak/állatok generálása
\item Térkép mentése/betöltése
\item Időjárás (eső,hó,köd)
\item Napszakok
\end{itemize}
