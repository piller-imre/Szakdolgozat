\chapter*{Összefoglalás}
\addcontentsline{toc}{chapter}{Összefoglalás}

A szakdolgozatom célja egy olyan procedurális generálási mód megvalósítása volt, ami változatos térképet generál izometrikus grafikájú játékokhoz. Az elméleti ismeretek bemutatása után az általam preferált technikák kiválasztásának indoklása és az implementálásukhoz szükséges részletek kifejtése történt meg, amit a program implementációjának bemutatása követett.

A szakdolgozatom készítése során sikerült létrehoznom egy olyan algoritmust amely a megadott bermeneti értékek alapján állít elő véletlenszerű értékekkel térképet.

A dolgozat egy központi eleme a különféle koordináta-rendszerek közötti transzformációk értelmezése és megvalósítása. A hexagonokból felépülő térképek kezelése a mezők indexelése és tárolási módja szempontjából rengeteg kérdést felvet, amelyre a dolgozat igykezett olyan szinten válaszokat adni, hogy azokat a játékfejlesztés során már fel lehessen használni.

\bigskip

A dolgozathoz készített alkalmazás már jelenlegi formájában is használható, vannak azonban még lehetőségek további fejlesztésekre. Az alábbiakban felvetés szintjén szerepel ezek közül néhány.
\begin{itemize}
\item A térképgenerátor optimalizálható a hatékonyabb memóriahasználat vagy rövidebb generálási idő érdekében.
\item A programkód egy része függ a Unity Engine-től. A későbbiekben célszerű lehet egy tisztán, Unity-től függetlenül használható C\# library-t létrehozni.
\item További objektumok (épületek/növények/utak/hidak/állatok) hozzáadásával színesebb, változatosabb térképeket érhetünk el, akár úgy, hogy a generálási módban lényegi változtatásokat nem végzünk.
\item Napszakok és további időjárásbeli tényezők (például eső, hó, köd) hozzáadásával szintén változatosabb lehet a generált térképek összképe.
\item A térképek lementésével és visszatöltésével egyszerűbben felhasználhatóvá válik a játékok számára a generátor használata.
\end{itemize}
