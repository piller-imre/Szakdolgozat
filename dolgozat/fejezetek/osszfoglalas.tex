\chapter*{Összefoglalás}
\addcontentsline{toc}{chapter}{Összefoglalás}

A szakdolgozatom célja egy olyan procedurális generálási mód megvalósítása volt, ami változatos térképet generál izometrikus grafikájú játékokhoz. Az elméleti ismeretek bemutatása után az általam preferált technikák kiválasztásának indoklása és az implementálásukhoz szükséges részletek kifejtése történt meg. Végezetül a program implementációjának bemutatása következett. A szakdolgozatom során igyekeztem ábrákkal segíteni az általam kevésbé egyértelműnek tűnő részeket. 

\bigskip

\noindent A szakdolgozatom készítése során sikerült létrehoznom egy olyan algoritmust amely a megadott bermeneti értkek alapján állít elő "véletlenszerű" térképet. A programomnak vannak még hiányosságai, amely további fejlesztést igényel.

\bigskip

\noindent Az alkalmazás további fejlesztésére különféle lehetőségek vannak. Az alábbiakban felvetés szintjén szerepel ezek közül néhány.
\begin{itemize}
\item A program optimalizálása különböző területeken (kód, teljesítmény).
\item További objektumok (épületek/növények/utak/hidak/állatok) hozzáadása, hogy sokszínűbb térképet kapjunk.
\item Lehetőség arra, hogy lementsük és betöltsük a térképet.
\item Napszakok és további időjárásbeli (eső,hó,köd) tényezők létrehozása, ezáltal változatosabb környezet kialakítása. 
\end{itemize}
