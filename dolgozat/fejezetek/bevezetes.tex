\Chapter{Bevezetés}

A szakdolgozatom célja egy olyan procedurális generálási mód kidolgozása, amellyel izometrikus grafikájú játékokhoz változatos térképeket lehet elõállítani. A térképek hexagon alapúak. A dolgozat azt a többlépcsõs folyamatot mutatja be, amely során a program a nagyobb összefüggõ egységek meghatározása után sorban határozza meg a különbözõ részletességi szinteken lévõ elemeket.

A szoftver C\# programozási nyelven készül. A C\# ismerete és a fejlesztéshez való alkalmazhatósága miatt esett a választásom erre a nyelvre, ezáltal a fejlesztéshez használt játékmotornak a Unity-t, fejlesztõ környezetnek pedig a Visual Studio-t választottam Windows 10 platformon.

