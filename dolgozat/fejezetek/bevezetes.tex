\chapter*{Bevezetés}
\addcontentsline{toc}{chapter}{Bevezetés}

A szakdolgozatom célja egy olyan procedurális generálási mód kidolgozása, amellyel izometrikus grafikájú játékokhoz változatos térképeket lehet előállítani.

Egy számítógépes játék megítélésénél nagyon fontos szerepet játszik a benne lévő tartalom (például textúrák, modellek, térképek) változatossága. A dolgozat alapvetően arra keres megoldást, hogy hogyan lehet meglévő elemekből, adott szabályszerűségek alapján természetesnek ható, a játékosok számára tetszetős térképeket automatikusan létrehozni.

A generált térképek hexagon alapúak. A dolgozat azt a többlépcsős folyamatot mutatja be, amely során a program a nagyobb összefüggő egységek meghatározása után sorban határozza meg a különböző részletességi szinteken lévő elemeket. Bemenetként a térkép fő jellemzőit, a generált térképpel szembeni elvárásokat kapja paraméterezésként a program. Bizonyos paraméterek szabályozhatóak a generálás után is (mint például a hőmérséklet), de a többségük nem. A térképnek a főbb paraméterei szabályozhatóak: méret, környezeti viszonyok, domborzat, folyók, növényzet, épületek. 

A hexagon alapú térképek mezőinek indexelésére, a hozzájuk rendelt adatok tárolási módjára nincs elfogadott, egységes megoldás. A dolgozat bemutatja a gyakrabban előforduló változatokat.

A számítógépes játékok lényeges elemei a textúrák és a modellek. A dolgozathoz készült alkalmazásban alacsony poligonszámú (\textit{low-poly}) modellek használatát választottam, mivel ez egy elterjedt stílusjegy, főként az indie játékok esetében.

A szoftver C\# programozási nyelven készül. A C\# ismerete és a fejlesztéshez való alkalmazhatósága miatt esett a választásom erre a nyelvre, ezáltal a fejlesztéshez használt játékmotornak a \textit{Unity}-t, fejlesztő környezetnek pedig a \textit{Visual Studio}-t választottam Windows 10 platformon. A program megtervezését követően a dolgozatban bemutatom a térkép generálására és megjelenítésére szolgáló alkalmazás implementációját. Ebben külön kitérek a térkép széleinek, a folyóknak, a hegyeknek és az épületeknek létrehozására, amelyek a térképek jellegzetes kialakítását eredményezik.
