\chapter*{Bevezetés}
\addcontentsline{toc}{chapter}{Bevezetés}

A szakdolgozatom célja egy olyan procedurális generálási mód kidolgozása, amellyel izometrikus grafikájú játékokhoz változatos térképeket lehet előállítani. A térképek hexagon alapúak. A dolgozat azt a többlépcsős folyamatot mutatja be, amely során a program a nagyobb összefüggő egységek meghatározása után sorban határozza meg a különböző részletességi szinteken lévő elemeket. Bemenetként a térkép fő jellemzőit, a generált térképpel szembeni elvárásokat kapja paraméterezésként a program. Bizonyos paraméterek szabályozhatóak a generálás után is (mint például a hőmérséklet), de a többségük nem. A térképnek a főbb paraméterei szabályozhatóak: méret, környezeti viszonyok, domborzat, folyók, növényzet, épületek. 

A szoftver C\# programozási nyelven készül. A C\# ismerete és a fejlesztéshez való alkalmazhatósága miatt esett a választásom erre a nyelvre, ezáltal a fejlesztéshez használt játékmotornak a Unity-t, fejlesztő környezetnek pedig a Visual Studio-t választottam Windows 10 platformon.

A dolgozat elején bemutatom az elméleti ismereteket, amik szükségesek egy hasonló alkalmazás elkészítéséhez, ezután a program implementációjának bemutatása következik.

% TODO: Érdemes lenne még néhány dolgot írni az eltervezett dolgokról, a megoldandó probléma fontosságáról.
